The project directory contains a folder named research paper. Inside this folder, there are three important files:

architecture_reference.md – a reference document that explains how the system architecture works. It is for understanding, not editing.

review_paper.tex – a completed review paper that has perfect LaTeX formatting and structure. This file is to be used strictly as a reference for formatting, structure, and tone.

final_paper.tex – this is the file that needs to be edited and updated. All changes and new writing must be made in this file only.

The goal is to build the final implementation paper for the proposed system in final_paper.tex.
This final paper should be based on and expand from the contents of the review_paper.tex, but with a focus on implementation and comparison.

Follow these steps carefully:

Understand the System:

The entire code base outside the research paper directory represents the proposed system for improving Zero Trust Architecture.

The system uses an Isolation Forest model and includes components for behavior tracking, anomaly detection, and dynamic trust scoring.

You may read through the code base to understand how the proposed system works, including its backend logic and AI model usage, but you must not modify any files other than final_paper.tex.

Purpose of the Final Paper:

The final paper is an implementation paper that builds on the concepts introduced in the review paper.

It should begin by comparing existing Zero Trust systems and identifying their flaws.

Then, it should introduce and explain the proposed system from the current code base as an improvement over existing systems.

Content Structure for final_paper.tex:
The paper should be structured in this logical flow:

Introduction: Define what Zero Trust Architecture (ZTA) is and briefly explain its importance.

Existing Systems Review: Use the review_paper.tex as a base to discuss existing systems, their design approaches, and their limitations.

Comparison and Flaws: Identify the flaws or inefficiencies of these existing ZTA implementations. Explain clearly what challenges or gaps exist.

Proposed System Overview: Introduce the proposed system from this project.

Explain the architecture and how it functions.

Describe the modules and interactions in the system.

Reference how the backend operates (using Isolation Forest, event tracking, trust scoring, anomaly detection, etc.).

Methodology and Model Explanation:

Discuss the methodology used for implementation.

Explain why the Isolation Forest model was chosen and how it is applied for anomaly detection.

Include explanation of how the model is trained and used in live monitoring.

Advantages and Improvements:

Clearly state how the proposed system overcomes the limitations of existing ZTA systems.

Emphasize improvements in adaptability, trust management, and automated anomaly handling.

Conclusion: Summarize the benefits of this architecture and why it presents a significant improvement for Zero Trust frameworks.

Formatting and Style Requirements:

The LaTeX formatting, structure, citation style, and presentation must exactly match the style used in review_paper.tex.

Follow the same font sizes, section hierarchy, figure formatting, paragraph spacing, and labeling conventions as seen in review_paper.tex.

The formatting of equations, tables, or diagrams should also be consistent with the review paper.

Do not alter formatting style or introduce new LaTeX templates or packages that were not used in the review paper.

Use the review_paper.tex as the standard for all stylistic and structural consistency.

Sources and Content Integration:

Use the review_paper.tex for content inspiration and references. You may reuse its conceptual parts where relevant, especially in describing existing systems and their limitations.

However, the proposed system section must be original, based on the architecture and implementation of the new system described in the project’s code base.

The architecture_reference.md file can be used for understanding the system, but content should be rewritten and refined appropriately for academic LaTeX presentation.

Editing Instructions:

Only edit the file final_paper.tex.

Do not alter or create new files.

Maintain the LaTeX integrity of the document (ensure compilable syntax).

Replace placeholder or draft text in final_paper.tex with the new structured and formatted content described above.

Ensure that all sections flow logically, are properly cited (if applicable), and align with the format from review_paper.tex.

Outcome Expectation:

The result should be a fully written and formatted implementation paper in LaTeX, suitable for publication.

It should demonstrate:

Understanding of Zero Trust Architecture fundamentals.

Comparison of existing ZTA models.

Explanation and technical justification of the proposed system (from the code base).

Clear presentation of methodology, models used, and improvements achieved.

The language should be formal, academic, and consistent with research paper writing standards.

The paper must compile without errors and maintain the stylistic consistency of the review paper.